\makeglossaries

\newglossaryentry{manet}{
  name=MANET,
  description={Mobile Ad-Hock Networks, un tipo de red mallada inalámbrica en la que todos los nodos se conectan directamente de manera directa, dinámica, y no jerárquica}
}

\newglossaryentry{LoRa}{
  name=LoRa,
  description={Long Range en inglés, una tecnología propietaria de comunicación por radio de baja frecuencia}
}

\newglossaryentry{STUN}{
  name=STUN,
  description={\emph{Session Traversal Utilities for NAT}, es una serie de técnicas estandarizadas, incluyendo un protocolo, para descubrir la presencia de NATs y conectar dispositivos directamente a través de ellas \cite{rfc3489} \cite{rfc5389}}
}

\newglossaryentry{NAT}{
  name=NAT,
  description={\emph{Network Address Translation}, es una técnica que permite a varios dispositivos compartir una dirección de red. Eric Rescorla tiene una muy buena descripción de los diferentes problemas que crean en su blog \cite{ekr-nat}}
}

\newglossaryentry{QUIC}{
  name=QUIC,
  description={Protocolo de transporte de red orientado a conexión, pero sobre UDP, lo cual consigue beneficios de rendimiento comparado con TCP. Gracias a que funciona sobre UDP, es más resistente a la pérdida de paquetes y por lo tanto es útil a la hora de hacer bypass de NATs y firewalls \cite{tailscale-nat-traversal} \cite{rfc9000}}
}

\newglossaryentry{TURN}{
  name=TURN,
  description={\emph{Traversal Using Relays around NAT} es un protocolo que asiste en aplicaciones cuando no se puede conectar directamente a través de las NAT, transmitiendo datos por un servidor intermedio \cite{rfc8656}}
}

\newglossaryentry{hole-punching}{
  name=hole punching,
  description={Técnica para establecer una conexión directa entre dos dispositivos a través de internet, donde estos dispositivos pueden estar detrás de firewalls o \gls{NAT}. Generalmente se usan protocolos como \gls{STUN}, \gls{TURN}, o QUIC address discovery \cite{quic-address-discovery}. TailScale tiene una genial explicación de cómo implementan NAT traversal para implementar su plataforma \cite{tailscale-nat-traversal}},
  see={NAT}
}

\newglossaryentry{CI/CD}{
  name=CI/CD,
  description={La Integración Continua/Entrega Continua (CI/CD, por sus
  siglas en inglés, \emph{Continuous Integration/Continuous Delivery}) es una
  práctica de desarrollo de software que automatiza el proceso
  (tradicionalmente manual) de llevar código desde su creación hasta su entrega
  en producción. Abarca las últimas fases del ciclo de vida del desarrollo de
  software, incluyendo la compilación, pruebas, implementación y despliegue de
  aplicaciones}
}

\newglossaryentry{memory-safe}{
  name=memory-safe,
  description={\emph{Memory safety} se refiere a la capacidad de un sistema o
  lenguaje de programación de prevenir problemas y vulnerabilidades de
  seguridad relacionadas con los accesos a memoria}
}

\newglossaryentry{thread-safe}{
  name=thread-safe,
  description={\emph{Thread safety} se refiere a la capacidad de un sistema de
  comportarse correctamente en presencia de accesos desde múltiples hilos
  concurrentemente}
}

\newglossaryentry{JVM}{
  name=JVM,
  description={La Máquina Virtual de Java (\emph{Java Virtual Machine} en
  inglés) es el entorno de ejecución que ejecuta el bytecode de Java para una
  plataforma específica}
}

\newglossaryentry{JNI}{
  name=JNI,
  description={La Interfaz Nativa de Java (\emph{Java Native Interface} en
  inglés) es una interfaz que permite interactuar a la \gls{JVM} con código
  nativo como el escrito en lenguajes como C, C++, o Rust}
}
 
\newglossaryentry{markdown}{
  name=Markdown,
  description={Lenguaje de marcado ligero con formato de texto plano diseñado para ser fácil de leer y escribir}
}

\newglossaryentry{transpilación}{
  name=Transpilación,
  description={Proceso mediante el cual se convierte código fuente de un
  lenguaje de programación a otro de un nivel de abstracción similar. Esto es
  en contraposición a la compilación, que convierte código de un lenguaje de
  programación de un nivel más alto a otro de un nivel más bajo}
}

\newglossaryentry{WPA}{
  name=WPA,
  description={\emph{WiFi Protected Access}, una serie de protocolos de
  seguridad estándares para proteger redes inalámbricas}
}

\newglossaryentry{wpa_supplicant}{
  name=wpa\_supplicant,
  description={Implementación de software libre de un suplicante WPA, utilizado
  para gestionar conexiones a redes WiFi en Linux, Android, y otros sistemas
  operativos}
}

\newglossaryentry{D-Bus}{
  name=D-Bus,
  description={Protocolo para la comunicación entre procesos usado en Linux}
}

\newglossaryentry{NDP}{
  name=NDP,
  description={\emph{Neighbor Discovery Protocol}, un protocolo de IPv6 para el
  descubrimiento de vecinos en la misma red},
}

\newglossaryentry{AWDL}{
  name=AWDL,
  description={\emph{Apple Wireless Device Link}, un protocolo propietario de
  Apple, similar en funcionalidad a WiFi Direct, usado para funcionalidad como
  \emph{AirDrop}},
}

\newglossaryentry{DMA}{
  name=DMA,
  description={\emph{Digital Markets Act}, regulación a nivel europeo para
  hacer los mercados digitales más competitivos},
}
\newglossaryentry{GNU}{
  name=GNU,
  description={\emph{GNU's Not Unix}, proyecto colaborativo iniciado por
  Richard Stallman en 1984 para desarrollar un sistema operativo libre y sus
  componentes necesarios},
}
\newglossaryentry{GPL}{
  name=GPL,
  description={\emph{GNU General Public License}, licencia copyleft del
  proyecto \Gls{GNU} que proporciona al usuario la libertad de usar, modificar,
  y redistribuir el software, con la condición de que trabajos derivados de él
  sean licenciados bajo los mismos términos}
}
\newglossaryentry{trait}{
  name=Trait,
  description={Colección de métodos, tipos asociados, etc independientes de la
  implementación en Rust. Similar a las interfaces en otros lenguajes de
  programación},
}
\newglossaryentry{MLS}{
  name=MLS,
  description={\emph{Messaging Layer Security}, un protocolo estándar
  \cite{rfc9420} de la IETF para el intercambio de claves en grupos de
  mensajería},
}

\newglossaryentry{ECDH}{
  name=ECDH,
  description={\emph{Elliptic Curve Diffie-Hellman}, una variante del protocolo
  Diffie-Hellman para establecer un secreto compartido via un canal inseguro
  que usa claves basadas en curvas elípticas},
}

\newglossaryentry{AES}{
  name=AES,
  description={\emph{Advanced Encryption Standard}, un algoritmo de cifrado
  simétrico ampliamente utilizado},
}

\newglossaryentry{GCM}{
  name=GCM,
  description={\emph{Galois/Counter Mode}, un modo de operación que, dado
  algoritmo de cifrado simétrico, permite cifrar datos y también verificarlos},
}

\newglossaryentry{MITM}{
  name=MITM,
  description={\emph{Man In The Middle}, o ataque de intermediario, un tipo de
  ataque en el que un atacante se interpone entre dos partes que están
  comunicándose, interceptando y posiblemente modificando la información que se
  intercambia sin que ninguna de las partes se dé cuenta},
}

\newglossaryentry{mpsc}{
  name=MPSC,
  description={\emph{Multiple Producer Single Consumer}, modelo de paso de
  mensajes donde uno o varios mensajeros envían (producen) mensajes, y un sólo
  consumidor los recibe.},
}
\newglossaryentry{kvm}{
  name=kvm,
  description={\emph{Kernel-based Virtual Machine}, solución de virtualización
  de Linux integrada en el núcleo.},
}
\newglossaryentry{ASan}{
  name=ASan,
  description={\emph{Address Sanitizer}, detector de errores de memoria para lenguajes compilados}
}
\newglossaryentry{WPS}{
  name=WPS,
  description={\emph{WiFi Protected Setup}, método de conexión a una red
  inalámbrica sin contraseña (via botón o pin)},
}
\glsaddallunused
