\makeglossaries

\newglossaryentry{manet}{
  name=MANET,
  description={Mobile Ad-Hock Networks, un tipo de red mallada inalámbrica en la que todos los nodos se conectan directamente de manera directa, dinámica, y no jerárquica}
}

\newglossaryentry{LoRa}{
  name=LoRa,
  description={Long Range en inglés, una tecnología propietaria de comunicación por radio de baja frecuencia}
}

\newglossaryentry{STUN}{
  name=STUN,
  description={\emph{Session Traversal Utilities for NAT}, es una serie de técnicas estandarizadas, incluyendo un protocolo, para descubrir la presencia de NATs y conectar dispositivos directamente a través de ellas \cite{rfc3489} \cite{rfc5389}}
}

\newglossaryentry{NAT}{
  name=NAT,
  description={\emph{Network Address Translation}, es una técnica que permite a varios dispositivos compartir una dirección de red. Eric Rescorla tiene una muy buena descripción de los diferentes problemas que crean en su blog \cite{ekr-nat}}
}

\newglossaryentry{QUIC}{
  name=QUIC,
  description={Protocolo de transporte de red orientado a conexión, pero sobre UDP, lo cual consigue beneficios de rendimiento comparado con TCP. Gracias a que funciona sobre UDP, es más resistente a la pérdida de paquetes y por lo tanto es útil a la hora de hacer bypass de NATs y firewalls \cite{tailscale-nat-traversal} \cite{rfc9000}}
}

\newglossaryentry{TURN}{
  name=TURN,
  description={\emph{Traversal Using Relays around NAT} es un protocolo que asiste en aplicaciones cuando no se puede conectar directamente a través de las NAT, transmitiendo datos por un servidor intermedio \cite{rfc8656}}
}

\newglossaryentry{hole-punching}{
  name=Hole punching,
  description={\emph{Hole punching} es una técnica para establecer una conexión directa entre dos dispositivos a través de internet, donde estos dispositivos pueden estar detrás de firewalls o \gls{NAT}. Generalmente se usan protocolos como \gls{STUN}, \gls{TURN}, o QUIC address discovery \cite{quic-address-discovery}. TailScale tiene una genial explicación de cómo implementan NAT traversal para implementar su plataforma \cite{tailscale-nat-traversal}}
}

\glsaddallunused
