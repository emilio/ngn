\chapter{Introducción}

La comunicación peer-to-peer (punto a punto, sin necesidad de infraestructura
centralizada), no son nada nuevo. Las \gls{manet} son comunes en entornos de
emergencia, tecnología militar, etc \footnote{\url{https://en.wikipedia.org/wiki/Wireless_ad_hoc_network\#Applications}}.

Esta tecnología es accesible a la mayoría de la población, via protocolos como
Bluetooth que usamos constantemente. Sin embargo este tipo de redes no son
comunes, a pesar de que las tarjetas de red inalámbrica de teléfonos y
ordenadores modernos soportan los protocolos necesarios para comunicación
directa por WiFi.

La principal hipótesis es que esto es porque usar este tipo de tecnologías en
hardware de consumo es difícil, por la complejidad de estos protocolos, que se
describirá en las siguientes secciones.

Por ello, se plantea la creación de una librería que abstraiga los detalles de
más bajo nivel (protocolo utilizado, establecimiento del enlace,
direccionamiento básico), para hacer más viable su adopción.
